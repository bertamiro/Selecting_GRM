% Template for PLoS
% Version 3.5 March 2018
%
% % % % % % % % % % % % % % % % % % % % % %
%
% -- IMPORTANT NOTE
%
% This template contains comments intended
% to minimize problems and delays during our production
% process. Please follow the template instructions
% whenever possible.
%
% % % % % % % % % % % % % % % % % % % % % % %
%
% Once your paper is accepted for publication,
% PLEASE REMOVE ALL TRACKED CHANGES in this file
% and leave only the final text of your manuscript.
% PLOS recommends the use of latexdiff to track changes during review, as this will help to maintain a clean tex file.
% Visit https://www.ctan.org/pkg/latexdiff?lang=en for info or contact us at latex@plos.org.
%
%
% There are no restrictions on package use within the LaTeX files except that
% no packages listed in the template may be deleted.
%
% Please do not include colors or graphics in the text.
%
% The manuscript LaTeX source should be contained within a single file (do not use \input, \externaldocument, or similar commands).
%
% % % % % % % % % % % % % % % % % % % % % % %
%
% -- FIGURES AND TABLES
%
% Please include tables/figure captions directly after the paragraph where they are first cited in the text.
%
% DO NOT INCLUDE GRAPHICS IN YOUR MANUSCRIPT
% - Figures should be uploaded separately from your manuscript file.
% - Figures generated using LaTeX should be extracted and removed from the PDF before submission.
% - Figures containing multiple panels/subfigures must be combined into one image file before submission.
% For figure citations, please use "Fig" instead of "Figure".
% See http://journals.plos.org/plosone/s/figures for PLOS figure guidelines.
%
% Tables should be cell-based and may not contain:
% - spacing/line breaks within cells to alter layout or alignment
% - do not nest tabular environments (no tabular environments within tabular environments)
% - no graphics or colored text (cell background color/shading OK)
% See http://journals.plos.org/plosone/s/tables for table guidelines.
%
% For tables that exceed the width of the text column, use the adjustwidth environment as illustrated in the example table in text below.
%
% % % % % % % % % % % % % % % % % % % % % % % %
%
% -- EQUATIONS, MATH SYMBOLS, SUBSCRIPTS, AND SUPERSCRIPTS
%
% IMPORTANT
% Below are a few tips to help format your equations and other special characters according to our specifications. For more tips to help reduce the possibility of formatting errors during conversion, please see our LaTeX guidelines at http://journals.plos.org/plosone/s/latex
%
% For inline equations, please be sure to include all portions of an equation in the math environment.
%
% Do not include text that is not math in the math environment.
%
% Please add line breaks to long display equations when possible in order to fit size of the column.
%
% For inline equations, please do not include punctuation (commas, etc) within the math environment unless this is part of the equation.
%
% When adding superscript or subscripts outside of brackets/braces, please group using {}.
%
% Do not use \cal for caligraphic font.  Instead, use \mathcal{}
%
% % % % % % % % % % % % % % % % % % % % % % % %
%
% Please contact latex@plos.org with any questions.
%
% % % % % % % % % % % % % % % % % % % % % % % %

\documentclass[10pt,letterpaper]{article}
\usepackage[top=0.85in,left=2.75in,footskip=0.75in]{geometry}

% amsmath and amssymb packages, useful for mathematical formulas and symbols
\usepackage{amsmath,amssymb}

% Use adjustwidth environment to exceed column width (see example table in text)
\usepackage{changepage}

% Use Unicode characters when possible
\usepackage[utf8x]{inputenc}

% textcomp package and marvosym package for additional characters
\usepackage{textcomp,marvosym}

% cite package, to clean up citations in the main text. Do not remove.
% \usepackage{cite}

% Use nameref to cite supporting information files (see Supporting Information section for more info)
\usepackage{nameref,hyperref}

% line numbers
\usepackage[right]{lineno}

% ligatures disabled
\usepackage{microtype}
\DisableLigatures[f]{encoding = *, family = * }

% color can be used to apply background shading to table cells only
\usepackage[table]{xcolor}

% array package and thick rules for tables
\usepackage{array}

% create "+" rule type for thick vertical lines
\newcolumntype{+}{!{\vrule width 2pt}}

% create \thickcline for thick horizontal lines of variable length
\newlength\savedwidth
\newcommand\thickcline[1]{%
  \noalign{\global\savedwidth\arrayrulewidth\global\arrayrulewidth 2pt}%
  \cline{#1}%
  \noalign{\vskip\arrayrulewidth}%
  \noalign{\global\arrayrulewidth\savedwidth}%
}

% \thickhline command for thick horizontal lines that span the table
\newcommand\thickhline{\noalign{\global\savedwidth\arrayrulewidth\global\arrayrulewidth 2pt}%
\hline
\noalign{\global\arrayrulewidth\savedwidth}}


% Remove comment for double spacing
%\usepackage{setspace}
%\doublespacing

% Text layout
\raggedright
\setlength{\parindent}{0.5cm}
\textwidth 5.25in
\textheight 8.75in

% Bold the 'Figure #' in the caption and separate it from the title/caption with a period
% Captions will be left justified
\usepackage[aboveskip=1pt,labelfont=bf,labelsep=period,justification=raggedright,singlelinecheck=off]{caption}
\renewcommand{\figurename}{Fig}

% Use the PLoS provided BiBTeX style
% \bibliographystyle{plos2015}

% Remove brackets from numbering in List of References
\makeatletter
\renewcommand{\@biblabel}[1]{\quad#1.}
\makeatother



% Header and Footer with logo
\usepackage{lastpage,fancyhdr,graphicx}
\usepackage{epstopdf}
%\pagestyle{myheadings}
\pagestyle{fancy}
\fancyhf{}
%\setlength{\headheight}{27.023pt}
%\lhead{\includegraphics[width=2.0in]{PLOS-submission.eps}}
\rfoot{\thepage/\pageref{LastPage}}
\renewcommand{\headrulewidth}{0pt}
\renewcommand{\footrule}{\hrule height 2pt \vspace{2mm}}
\fancyheadoffset[L]{2.25in}
\fancyfootoffset[L]{2.25in}
\lfoot{\today}

%% Include all macros below

\newcommand{\lorem}{{\bf LOREM}}
\newcommand{\ipsum}{{\bf IPSUM}}






\usepackage{forarray}
\usepackage{xstring}
\newcommand{\getIndex}[2]{
  \ForEach{,}{\IfEq{#1}{\thislevelitem}{\number\thislevelcount\ExitForEach}{}}{#2}
}

\setcounter{secnumdepth}{0}

\newcommand{\getAff}[1]{
  \getIndex{#1}{Universitat de Barcelona,International Rice Research Institute}
}

\providecommand{\tightlist}{%
  \setlength{\itemsep}{0pt}\setlength{\parskip}{0pt}}

\begin{document}
\vspace*{0.2in}

% Title must be 250 characters or less.
\begin{flushleft}
{\Large
\textbf\newline{\emph{A heuristic algorithm to select genes potentially regulated by
methylation}} % Please use "sentence case" for title and headings (capitalize only the first word in a title (or heading), the first word in a subtitle (or subheading), and any proper nouns).
}
\newline
% Insert author names, affiliations and corresponding author email (do not include titles, positions, or degrees).
\\
Alex Sanchez-Pla\textsuperscript{\getAff{Universitat de Barcelona, (UB), Spain}}\textsuperscript{*},
Berta Miro\textsuperscript{\getAff{International Rice Research Institute, (IRRI), Philippines}}\\
\bigskip
\textbf{\getAff{Universitat de Barcelona}}Departament of Genetics Microbiology and Statistics, Avda Diagonal 645,
Barcelona, 08028\\
\textbf{\getAff{International Rice Research Institute}}Department, Street, City, State, Zip\\
\bigskip
* Corresponding author: asanchez@ub.edu\\
\end{flushleft}
% Please keep the abstract below 300 words
\section*{Abstract}
Methylation is a key process in cancer. Usually it acts by inhibiting
the expression of the gene but if methylation is low then any values of
expression, high or low, can be found. This suggests that to select
genes regulated by methylation one may look for patterns in the relation
between gene expression and methylation showing either an L-shape or
negative correlation between expression and methylation. We have
developed a heuristic algorithm that mimics the process of visually
selecting an ``L-shape'', that is genes that can show a wide range of
expression values (low to high) when methylation is low, but only low
expressions for intermediate or high methylation. We have compared the
method with naïve correlation and, despite not being able to quantify
its accuracy -because no dataset with ``TRUE'' L-shaped genes is
available- its performance seems to be very good especially due to its
flexibility. The method has been implemented in an R package,
``Lheuristic'' and a Shiny application, both available from GitHub
(http://github.com/alexsanchezpla). Given two matrices -expression and
methylation values - with the same row and column names the program
offers the possibility to select genes based on either negative
correlation, the heuristic algorithm or both methods at once. Once genes
have been selected, results can be interactively reviewed, plotted or
downloaded.

% Please keep the Author Summary between 150 and 200 words
% Use first person. PLOS ONE authors please skip this step.
% Author Summary not valid for PLOS ONE submissions.

\linenumbers

% Use "Eq" instead of "Equation" for equation citations.
\hypertarget{introduction-and-background}{%
\section{Introduction and
Background}\label{introduction-and-background}}

\hypertarget{introduction-to-methylation}{%
\subsection{Introduction to
methylation}\label{introduction-to-methylation}}

Epigenetic marks modulate gene expression without affecting the DNA
nucleotide sequence. These potentially heritable changes are, for
example, DNA methylation or histone acetylation ({[}1{]}). DNA
methylation is the most studied epigenetic process in humans. The
process is based on the addition of a methyl group, mostly in CpG
dinucleotides. The CpG dinucleotides tend to group in areas of less than
500kb and with higher than 55\% C and G content , these regions are
named islands; further from the island the region is called shore and
further from the shore it is called shelf. More than 60\% of promoter
regions are associated with CpG islands ({[}2{]}) and the methylation of
these is linked to gene silencing and gene expression inhibition. DNA
methylation has been linked to the regulation of numerous cellular
processes, including embryonic development, or X-chromosome inactivation
and preservation of chromosome stability among others. DNA methylation
has also been observed in autoimmune diseases, metabolic disorders,
neurological disorders, and other processes that despite being natural
they are debilitating, like ageing for example; and it can also be
correlated with drug or treatment response ({[}3{]}; {[}4{]}; {[}5{]};
{[}6{]}). Most research on this area has been, however, focused on tumor
repressor genes, which are often silenced in cancer cells due to
hypermethylation. This is an important mechanism of gene silencing
during tumor progression ({[}7{]}). On the contrary, a general level of
hypomethylation has been observed in human tumors ({[}8{]}); therefore,
hypomethylation is a useful mechanism to distinguish genes of some human
cancers from their normal counterparts.\\
In the human genome, about 80\% of cytosines in the 56 million CpG sites
are methylated to 5-methylcytosines. The methylation pattern of DNA is
highly variable among cells types and developmental stages and
influenced by disease processes and genetic factors. The relationship
between gene expression and methylation has been associated with cancer
and extensively studied, therefore it has produced fruitful results
({[}9{]}).

\hypertarget{analysis-of-genes-regulatated-by-methylation}{%
\subsection{Analysis of genes regulatated by
methylation}\label{analysis-of-genes-regulatated-by-methylation}}

With the abundance of emerging evidence indicating the important role of
DNA methylation in common diseases, researchers have attempted to use
DNA methylation as a biomarker to identify epigenetic changes that are
associated with disease status.~While the genetic events that drive the
tumorigenic process are relatively well characterized for colorectal
cancer, the epigenetic events and their impact on the transcriptional
reprogramming observed in colorectal tumors have not been extensively
characterized. Although recent genome-wide studies have analyzed the
genomic distribution of hypermethylated CpGs in a small number of
colorectal tumors (ref), a detailed analysis of the subset of these
events that are important for gene expression regulation is currently
lacking. Just as gene expression microarrays accelerated and
revolutionized the study of transcriptional regulation, rapidly
improving technologies are increasingly enabling researchers to assess
locus-specific DNA methylation on a genome-wide scale. Recently various
high-throughput approaches based on bisulfite conversion combined with
next generation sequencing have been developed and applied for the
genome wide analysis of DNA methylation. These methods provide single
base pair resolution, quantitative DNA methylation data with genome wide
coverage. There are various experimental types of methylation assays,
but overall, methylation levels can be represented in one of three
types: discrete, continuous or categorical. Therefore, methylation can
be quantified by directly using read count information , ratio data
(which may lose biological variability) or both. Once the DNA samples
are processed, an important issue to be considered is the influence of
the statistical analysis on the accuracy of the genomic methylation
level estimation from bisulfite sequencing data. The accuracy of the
statistical approach to methylation quantification increases with the
sequencing depth of the particular cytosine residue ({[}10{]}). However,
there are regression and neighboring analysis techniques that can
counteract the lack of sequence depth in a particular CpG ({[}11{]}).

\hypertarget{existing-methods-and-analyses}{%
\subsubsection{Existing methods and
analyses}\label{existing-methods-and-analyses}}

The association between gene expression and DNA methylation in the CpG
islands in particular has been long studied; and as a result, mostly
negative correlations have been found to relate to cancer driven
mechanisms (\textbf{???}), but this inverse relationship between DNA
methylation of the first intron in particular and gene expression is a
broad mechanism to down-regulate gene expression and it is found in
numerous processes, organisms and tissues ({[}12{]}). There have been
various studies analysing this correlation using various approaches. For
example, Massie et al., (2017) looked at the relationship between gene
expression and DNA methylation at the probe level rather than at the
gene level. They narrowed a list of genes regulated by methylation that
were identified in more than 3 out of 17 studies. Another study analysed
the TCGA database to identify patterns in DNA CpG methylation and gene
expression and tumor status. They found that the association involved a
reduced number of genes linked to cancer than originally anticipated
(around the hundreds) and that not all correlations were negative
({[}13{]}). Another recent paper reported two different models for
analysis of DNA methylation and regulation of gene expression, one for
negatively correlated genes and one for positively correlated genes
(Klett et al., 2018). They used expression (GSE106582) and methylation
datasets (GSE101764) containing 194 samples, 77 tumors and 117 of the
mucose. By random forest analysis they were able to classify genes into
cancer related and not related. Still methodologies to find tune
classification into cancer/disease related and not cancer/disease
related are still needed. A previously developed method was the
selection of genes with an L-shape association between the expression
and the methylation datasets (Sanchez-Pla et al., 2015). In this
research, they focused on the CMI and on a method based on spline
regression. They observed that the first method would detect L-shaped
genes more accurately in big datasets. On the other hand, the splines
clustering was not size dependent, but it would yield a smaller number
of samples. Other research exists that aimed to identify genes regulated
by methylation according to the expression methylation patterns;
however, they only use a particular methodology like the CMI
(\textbf{???}) with positive results. A paper focused on the
identification of genes regulated by methylation through unsupervised
clustering techniques to identify CRC subtypes was able to confirm
existing subtypes ({[}14{]}).There has been other work that focused on
the development of platforms for the identification of genes regulated
by methylation. One of these packages is MEXPRESS (Koch et al., 2017).
This package has a web interface that allows the user to visualize
expression and methylation data from genes in the TCGA data. The
visualization collocates for each selected gene, CpG islands, with
transcripts expression together with other clinical values such as
gender and age. The tool also generates p-values in relation to the
variables specified. Another one of these packages is Methylmix (ref).
The algorithm is based on a beta mixture model that identifies
methylation states and compares them with what they call normal
conditions to find hypo- and hyper-methylated genes. They developed a
new statistic coeficient, the Differential Methylation value or DM-value
which is defined as the difference of a methylation state with the
normal methylation state. Then, they correlate that coefficient with
gene expression data to characterize the association between methylation
level and gene expression. For expression and methylation correlation
analyses of both RNA and DNA molecules there is also a web based tool
that analyses methylated genes based on TCGA data, called MethHC
(http://methhc.mbc.nctu.edu.tw/php/search.php?opt=gene, Huang et al.,
2015). This database has an analysis tool that provides gene-specific
analysis for various diseases, and the information is displayed as a
comparison between diseased and normal (non-diseased) conditions; list
of highest and lowest methylated (hyper and hypo) genes; as well as
correlations between expression and methylation. In this, methylation is
a binary value (0,1) Other methodologies to identify methylated genes
associated with cancer is through text mining analysis, as in the
PubMeth database (www.pubmeth.org, {[}15{]}). In this, they identified
5000 genes of 1000 publications. However, high-throughput methodologies
that offer an impartial approach to the identification of genes
regulated by methylation still need further development and fine-tuning.
Here we present such a methodology that will select, out of a gene
expression and DNA methylation subsets, those genes that present a
negative correlation, and are therefore regulated by methylation. The
L-shaped heuristic method to identify genes regulated by methylation was
tested and tuned for experimental expression and methylation paired
datasets after normalization using other standard methods.

\hypertarget{material-and-methods}{%
\section{Material and Methods}\label{material-and-methods}}

As we have described in the previous section, although there are various
approaches to selecting genes based on the relationship between
methylation levels and gene expression, none of them are completely
satisfactory.

In this section we present the method we have developed to select genes
in which the pattern of the relationship is ``L-shaped.'' In fact,
taking biological processes into account, this is a very common and very
reasonably expected pattern when genes are regulated by methylation.
Furthermore, as we will see later, it is not only important but can be
partially missed by ``naïve'' methods such as significant negative
correlation, which increases the interest of our proposal.

\hypertarget{rationale-of-the-approach}{%
\subsection{Rationale of the approach}\label{rationale-of-the-approach}}

After trying different approaches to detect L-shapes, one often comes
back to an intuïtive idea: If we are looking for genes whose expression
can take any value when methylation is low, and tends to decrease as
methylation increases one should observe that points in the
methylation-expression scatterplot tend to be scattered near the
vertical and horizontal positive axes showing an L-shape. If this does
not happen genes can be found anywhere in the scatterplot and we can
call it a non-L-shape. That is:

\begin{itemize}
\tightlist
\item
  The more the points cluster near the vertical and horizontal axes, the
  more L-shaped can be considered the scatterplot.
\item
  The more the points move away from the axes, the least L-shaped the
  scatterplot is.
\end{itemize}

Figure 1 illustrates these two possibilities in two real but
non-identified genes.

\hypertarget{an-algorithm-to-select-l-shape-scatterplots}{%
\subsection{An algorithm to select L-shape
scatterplots}\label{an-algorithm-to-select-l-shape-scatterplots}}

The intuitive idea presented above can be made more explicit by
introducing a ``three-band rule'' as follows:

\begin{enumerate}
\item Overimpose a $3\times 3$ grid on the scatterplot.
\item Classify the scatterplot as \textbf{``L'' or ``non-L''} based on a small set of conditions:
\begin{enumerate}
  \item There must be a \emph{minimum} number of points in the upper-left (cell (1,1)) and lower right (cell (3,3)) corners of the grid.
  \item There must be a \emph{maximum} number of points in the upper right (cell (1,3)) because points there mean hypermethylation and hyperexpression which is the opposite of what we are looking for.
  \item We will usually \emph{not require to have a minimum of points in cell (3,1)} unless we are really willing to have an L-shape (in our setting we will also be happy tho recover diagonals, which also reflect a negative correlation!).
\end{enumerate}
\item Score points on each subgrid in such a way that
\begin{enumerate}
    \item Points in permitted regions (left-outer margin, i.e. cells: (1,1), (2,2), (3,1), (3,2), (3,3)) score positively if the scatterplot has been classified as L or zero if it has been classified as non-L.
    \item Points in non-desired regions (outer band. i.e. cells (1,2), (1,3), (2,3)) score negatively in all cases.
    \item Some regions may be declared neutral and not-score, such as cell (2,2).
\end{enumerate}
\item Use cross-validation to tune scoring parameters (\textit{if a set of positive and negative L-shaped genes is available}). 
\end{enumerate}

The previous scheme can be summarized using the following equation.
\begin{equation}
S(X) = W_L \circ X \times \mathbf{1}_L(X) + W_{L^C} \circ X \times \mathbf{1}_{L^c}(X),
\end{equation} where

\begin{itemize}
\item ${X}$ is the matrix of \emph{counts}, i.e. the number of counts in each cell of the grid,
\item ${W_L}$ is the matrix of scores per cell and point \emph{if the scatterplot has been classified as $L$},
\item ${W_{L^c}}$ is the matrix of scores per cell and point \emph{if the scatterplot has been classified as non-$L$ ($L^c$)},
\end{itemize}

and \(\circ\) represents the hadamard product of the two matrices
\(W_{L/L^c}\) (i.e.~elementwise multiplication of the two matrices) and
\(\mathbf{1}_{L/L^c}()\) is the indicator function for \(L\) or \(L^c\).

The fact that the scatterplot is assigned to \(L\) or \(L^c\) can also
be described as the hadamard product of three matrices: \begin{equation}
\mathbf{1}_L(X) = \bigwedge_{i,j} X \circ C \circ \left( mMP \times \sum_{i,j}x_{ij}\right),
\end{equation} where

\begin{itemize}
\item ${X}$ is the matrix of \emph{counts}, i.e. the number of counts in each cell of the grid,
\item $C$ is the matrix of conditions to be verified \emph{if the scatterplot has to be classified as $L$},
\item $mMP$ is the matrix of minimum and Maximum Percentages of points to have in each cell \emph{if the scatterplot has to be classified as $L$},
\item $\circ$ represents the pointwise logical operation which allows that the product of the three cells becomes a logical operation and
\item $\bigwedge_{i,j}$ represents an logical ``AND'' operation of all cells, that is if all cells are TRUE the result is assigned to $L$ and if one fails it is assigned to $L^c$.
\end{itemize}

\hypertarget{on-the-sensitivity-and-specificity-of-the-method}{%
\subsection{On the sensitivity and specificity of the
method}\label{on-the-sensitivity-and-specificity-of-the-method}}

\hypertarget{tuning-the-algorithms-parameters}{%
\subsubsection{Tuning the algorithm's
parameters}\label{tuning-the-algorithms-parameters}}

\hypertarget{synthetic-dataset-generation-for-the-simulation-studies}{%
\subsection{Synthetic dataset generation for the simulation
studies}\label{synthetic-dataset-generation-for-the-simulation-studies}}

The R package \texttt{simstudy} was used to create 4 artificial datasets
by using the splines method
(https://cran.r-project.org/web/packages/simstudy/simstudy.pdf). The
package allows for designing data points on a pre-defined spline, in
which knots, limits and dispersion can be tuned. The splines are
generated based on a fixed X variable representing the methylation
values. The artificial datasets contained a total of 1000 genes, and the
data points were developed based on 2 parameters with 2 levels each. The
first parameter was the number of samples and the second the \% of true
regulated by methylation genes that a sample would contain (with an
expression by methylation scatterplot or spline following an L-shape).
The number of samples considered was of 50 and 1000, and the \% of true
methylated genes in each dataset was 1\% and 10\%. Additionally, the
shape of the negative genes (not regulated by methylation) was also
pre-defined and classified into 3 different scatterplot shapes (Figure
XXX) and the percentage of genes in each category equaled to 1/3 prior
subtraction of the true GRM genes.

\hypertarget{simulation-of-heuristic-classification-with-the-synthetic-datasets}{%
\subsection{Simulation of heuristic classification with the synthetic
datasets}\label{simulation-of-heuristic-classification-with-the-synthetic-datasets}}

The heuristic method was tested with the 4 artificial datasets: 50
samples, 1\% of GRM genes; 50 samples, 10\% GRM genes; 1000 samples, 1\%
GRM genes; and 1000 samples, 10\% GRM genes. After running the model,
sensitivity, specificity, and accuracy were measured and compared
between datsets.

\hypertarget{results}{%
\section{Results}\label{results}}

\hypertarget{measures-of-performance-for-the-heuristic-method-with-synthetic-datasets}{%
\subsection{Measures of performance for the heuristic method with
synthetic
datasets}\label{measures-of-performance-for-the-heuristic-method-with-synthetic-datasets}}

Sensitivity, specificity and accuracy for the heuristic model were
measured for the 4 synthetic datsets (Figure {[}ssa{]}) with predefined
parameters described in the above section. Specificity was the parameter
that scored highest in all datsets, with values between 0.99 (for the
datasets with 50 samples) and 1 (for the datasets with 1000 samples).
The second highest parameter was accuracy. In this, both datasets
containing 1\% of GRM genes scored 0.99, whereas the datasets with 10\%
of GRM genes scored 0.93 (for the one with 50 samples) and 0.92 (for the
one with 1000 samples). Finally, the sensitivity values were the lowest
in the combination of 1\% GRM and 1000 samples (0.1) and highest for 1\%
GRM and 50 samples (0.5). These results indicate that the classification
scored better non-L shaped scatterplots (true negatives) than L-shaped
scatterplots (true positives).

\hypertarget{discussion}{%
\section{Discussion}\label{discussion}}

\hypertarget{supporting-information}{%
\section{Supporting information}\label{supporting-information}}

\hypertarget{references}{%
\section*{References}\label{references}}
\addcontentsline{toc}{section}{References}

\clearpage

\hypertarget{figures-and-captions}{%
\section{Figures and captions}\label{figures-and-captions}}

\begin{figure}
\hypertarget{id}{%
\centering
\includegraphics[width=0.5\textwidth,height=0.5\textheight]{figures/nonLshapeVSLshape1.png}
\caption{Example of non-Lshape vs L-shape for the
methylation--expression scatterplots associated with two fictitial
genes}\label{id}
}
\end{figure}

\hypertarget{refs}{}
\leavevmode\hypertarget{ref-Berger2009}{}%
1. Berger SL, Kouzarides T, Shiekhattar R, Shilatifard A. An operational
definition of epigenetics. Genes and Development. 2009;23: 781--783.
doi:\href{https://doi.org/10.1101/gad.1787609}{10.1101/gad.1787609}

\leavevmode\hypertarget{ref-Saxonov2006}{}%
2. Saxonov S, Berg P, Brutlag DL. A genome-wide analysis of CpG
dinucleotides in the human genome distinguishes two distinct classes of
promoters. Proceedings of the National Academy of Sciences of the United
States of America. 2006;103: 1412--1417.
doi:\href{https://doi.org/10.1073/pnas.0510310103}{10.1073/pnas.0510310103}

\leavevmode\hypertarget{ref-Reik2007}{}%
3. Reik W. Stability and flexibility of epigenetic gene regulation in
mammalian development. Nature Publishing Group; 2007. pp. 425--432.
doi:\href{https://doi.org/10.1038/nature05918}{10.1038/nature05918}

\leavevmode\hypertarget{ref-Portela2010}{}%
4. Portela A, Esteller M. Epigenetic modifications and human disease.
2010. pp. 1057--1068.
doi:\href{https://doi.org/10.1038/nbt.1685}{10.1038/nbt.1685}

\leavevmode\hypertarget{ref-Feil2012}{}%
5. Feil R, Fraga MF. Epigenetics and the environment: Emerging patterns
and implications. 2012. pp. 97--109.
doi:\href{https://doi.org/10.1038/nrg3142}{10.1038/nrg3142}

\leavevmode\hypertarget{ref-Benayoun2015}{}%
6. Benayoun BA, Pollina EA, Brunet A. Epigenetic regulation of ageing:
Linking environmental inputs to genomic stability. Nature Publishing
Group; 2015. pp. 593--610.
doi:\href{https://doi.org/10.1038/nrm4048}{10.1038/nrm4048}

\leavevmode\hypertarget{ref-Jones2002}{}%
7. Jones PA, Baylin SB. The fundamental role of epigenetic events in
cancer. 2002. pp. 415--428.
doi:\href{https://doi.org/10.1038/nrg816}{10.1038/nrg816}

\leavevmode\hypertarget{ref-Feinberg1983}{}%
8. Feinberg AP, Vogelstein B. Hypomethylation distinguishes genes of
some human cancers from their normal counterparts. Nature. 1983;301:
89--92. doi:\href{https://doi.org/10.1038/301089a0}{10.1038/301089a0}

\leavevmode\hypertarget{ref-Yang2014}{}%
9. Yang X, Han H, DeCarvalho DD, Lay FD, Jones PA, Liang G. Gene body
methylation can alter gene expression and is a therapeutic target in
cancer. Cancer Cell. Cell Press; 2014;26: 577--590.
doi:\href{https://doi.org/10.1016/j.ccr.2014.07.028}{10.1016/j.ccr.2014.07.028}

\leavevmode\hypertarget{ref-Zhang2010}{}%
10. Zhang Y, Jeltsch A. The application of next generation sequencing in
DNA methylation analysis. 2010. pp. 85--101.
doi:\href{https://doi.org/10.3390/genes1010085}{10.3390/genes1010085}

\leavevmode\hypertarget{ref-Wreczycka2017}{}%
11. Wreczycka K, Gosdschan A, Yusuf D, Grüning B, Assenov Y, Akalin A.
Strategies for analyzing bisulfite sequencing data. Journal of
Biotechnology. 2017;261: 105--115.
doi:\href{https://doi.org/10.1016/j.jbiotec.2017.08.007}{10.1016/j.jbiotec.2017.08.007}

\leavevmode\hypertarget{ref-Anastasiadi2018}{}%
12. Anastasiadi D, Esteve-Codina A, Piferrer F. Consistent inverse
correlation between DNA methylation of the first intron and gene
expression across tissues and species. Epigenetics and Chromatin. BioMed
Central Ltd. 2018;11.
doi:\href{https://doi.org/10.1186/s13072-018-0205-1}{10.1186/s13072-018-0205-1}

\leavevmode\hypertarget{ref-Long2017}{}%
13. Long MD, Smiraglia DJ, Campbell MJ. The Genomic Impact of DNA CpG
Methylation on Gene Expression; Relationships in Prostate Cancer.
Biomolecules. Multidisciplinary Digital Publishing Institute (MDPI);
2017;7.
doi:\href{https://doi.org/10.3390/biom7010015}{10.3390/biom7010015}

\leavevmode\hypertarget{ref-Barat2015}{}%
14. Barat A, Ruskin H, Byrne A, Prehn J. Integrating Colon Cancer
Microarray Data: Associating Locus-Specific Methylation Groups to Gene
Expression-Based Classifications. Microarrays. MDPI AG; 2015;4:
630--646.
doi:\href{https://doi.org/10.3390/microarrays4040630}{10.3390/microarrays4040630}

\leavevmode\hypertarget{ref-Ongenaert2007}{}%
15. Ongenaert M, Van Neste L, De Meyer T, Menschaert G, Bekaert S, Van
Criekinge W. PubMeth: a cancer methylation database combining
text-mining and expert annotation. Nucleic Acids Research. 2007;36:
D842--D846.
doi:\href{https://doi.org/10.1093/nar/gkm788}{10.1093/nar/gkm788}

\nolinenumbers


\end{document}

